\section*{Abstract}
Containers, such as Docker and LinuxKit, are widely used because of the perceived isolation and security they provide 
against the potentially malicious or buggy user programs running inside of them. However, this perception may be false. 
It is possible to trigger zero-day kernel bugs from inside of a container, which could lead to such security problems as privilege escalation. 
The core of this problem lies in that these containers are allowed access to rarely executed paths in the host operating system kernel, 
which often contain bugs. A previous study \cite{Lock-in-Pop} has proven the value of limiting kernel access to 
only the frequently used ``popular paths,'' as they contain fewer security bugs. 
Therefore, if an application could run using only these paths, the security of the container would be greatly enhanced. 

In this paper, we show that if the popular paths metric is applied to existing containers, such as LinuxKit, 
it can create the true isolation its users desire with no loss in performance. 
We start by presenting a method to systematically identify the popular paths for a LinuxKit virtual machine 
by collecting kernel trace data from 200 popular Docker container images from Docker Hub. 
We found that more than 95\% of the kernel trace overlapped, making it possible to identify the popular paths by running 
only the first 10 containers, rather than sampling hundreds of application containers. 
We also verified that, of 50 Linux kernel CVE security vulnerabilities only about 5\% of were present in the popular paths. 
With this extracted road map, we were able to implement a system to automatically instrument the kernel in order to 
prevent container applications from accessing risky unpopular paths. 
We call our modified kernel the ``security logging mode'' kernel. 
This ``security logging mode'' kernel stops execution of risky code, and thus prevents dangerous application behaviors. 
It will also warn users about which container application are trying to access which unpopular paths, 
which can be used to identify suspicious and malicious applications. 
Lastly, we demonstrated that the 200 popular Docker containers were still able to run their normal workload 
using the ``security logging mode'' kernel, and security was not improved at the cost of efficiency, 
as space overhead and runtime performance overhead incurred only xx\%, and xx\%, respectively.  