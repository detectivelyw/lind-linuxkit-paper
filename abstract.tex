\section*{Abstract}
Containers, such as Docker and LinuxKit, are widely used because of the perceived isolation and security 
they provide against the potentially malicious or buggy user programs running inside of them. 
However, this perception may be false. It is possible to trigger zero-day kernel bugs from inside of a container, 
which could lead to such security problems  as privilege escalation.  
The reason for this vulnerability is that these containers are allowed access to rarely executed paths in the host operating system kernel, 
which often contain bugs. A previous study \cite{Lock-in-Pop} has proven the value of limiting kernel access to 
only the frequently used ``popular paths,'' as they contain fewer security bugs. 
Therefore, if an application could run using only these paths, the security of the container would be greatly enhanced. 

In this paper, we show that the popular paths metric can be used to improve the security of existing containers, such as LinuxKit. 
We designed and implemented a \textbf{\textit{Secure Logging System}}, which first finds the popular paths for the LinuxKit kernel 
by profiling popular Docker containers from Docker Hub \cite{DockerHub}, 
and then creating a \textbf{\textit{secure logging kernel}} designed to run 
only on those secure paths. Whenever a line of code from an unpopular path is to be executed, this kernel generates warning messages 
to alert users about potential security breaches. 
Our evaluation verified that the containers tested were able to run their default workload normally using our secure logging kernel, 
touching the popular paths 99\% of the time. 
As an examination of 50 Linux kernel CVE security vulnerabilities confirmed only 6\% of them were present in the popular paths, 
this means our design is a more inherently secure alternative to conventional containers. 
Furthermore, as our performance evaluation shows, running Docker containers using our secure logging kernel 
only incurs about 2\% runtime overhead, and around 0.5\% extra cost of memory space. 
Therefore, the new design can provide better security without any significant loss of efficiency.