\section*{Abstract}
One of several reasons why containers, such as Docker and LinuxKit, are so widely used is because of the perceived isolation and security they provide against 
the potentially malicious or buggy user programs running inside of them. However, containers provide no protection from the zero-day kernel bugs inside the kernel itself, 
which, if triggered, could lead to such security problems as privilege escalation. 
Containers remain vulnerable because they allow access to rarely executed paths where such bugs can be found. 
Limiting kernel access to only frequently used \textbf{\textit{popular paths}}, which have previously been proven to contain fewer security bugs, would greatly reduce this risk. 
Therefore, if a container could run using only these paths, its security would be greatly enhanced. 

In this paper, we propose a method to enhance the security of existing container designs by leveraging the popular paths metric. 
We designed and implemented a \textbf{\textit{Secure Logging System}}, which first systematically identifies popular paths data in widely-used containers, 
and uses this data to create a kernel that generates security logs and warning messages when a potentially risky path is reached. 
Users can then decide whether the potentially dangerous code execution should be stopped. We tested this modified kernel, which we named the \textbf{\textit{Secure Logging Kernel}}, 
to determine its  ability to run containers. Our evaluation verified that, more than 99.9\% of the time, containers were able to run their default workload using just the popular paths. 
Furthermore, only 6\% of 50 Linux kernel CVE security vulnerabilities we examined were present in the popular paths on which these containers ran, 
greatly reducing any potential risk of triggering kernel bugs. 
Lastly, our performance evaluation shows that containers running our Secure Logging Kernel only incur about 0.1\% runtime overhead on average, 
and around 0.37\% extra cost of memory space. This means the Secure Logging Kernel can improve security with little or no loss of efficiency. 