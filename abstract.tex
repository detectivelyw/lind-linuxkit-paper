\section*{Abstract}
One reason why containers, such as Docker and LinuxKit, are so widely used is because of the perceived isolation and security they provide against 
the potentially malicious or buggy user programs running inside of them. Yet, most security measures designed for containers do not take into account 
how to protect against the zero-day kernel bugs inside the kernel itself. 
Containers are vulnerable to these bugs because they allow access to rarely executed paths where such vulnerabilities can be found. 
Limiting kernel access to only frequently used \textbf{\textit{popular paths}}, which have previously been proven to contain fewer security bugs, 
would greatly reduce the risk of triggering these vulnerabilities. 

In this paper, we present a multi-step methodology for improving container security by leveraging data about popular paths into a kernel tailoring strategy. 
It starts with a systematic approach to identifying and capturing the popular paths data for widely-used container applications. 
By evaluating this data at different levels of granularity (line, function, and file), users can get fresh insights into which parts of the kernel are being used, 
and so make informed decisions about which code is safe to execute. 
The data also guided the implementation of a tailored kernel called the \textbf{\textit{UnPopular Action Kernel (UnPAK)}}. 
\textbf{UnPAK} registers when a potentially risky path is reached and can respond with a number of actions, from issuing warning messages to denying execution of commands. 
When testing containers using UnPAK, we found that they were able to run their default workload using just the popular paths more than 99.9\% of the time. 
Furthermore, the kernel was able to operate with only minor increments (0.1\% on average) in runtime overhead, and only around 0.37\% extra cost in memory space. 
Most importantly, since only 6\% of 50 Linux kernel CVE security vulnerabilities we examined were present in the popular paths, 
utilizing the UnPAK system offers a way to reduce the risk of triggering bugs without sacrificing operational efficiency. 
Lastly, UnPAK compares favorably with three other current kernel tailoring strategies, as it is more adaptable to a wider variety of applications, and can reduce the largest amount of attack surface. 