\section*{Abstract}
Containers, such as Docker and LinuxKit, are widely used because of the perceived isolation and security 
they provide against the potentially malicious or buggy user programs running inside of them. 
However, it is possible to trigger zero-day kernel bugs from inside of a container, which could lead to such security problems as privilege escalation. 
Containers remain vulnerable because they are allowed access to rarely executed \textbf{\textit{unpopular paths}} in the host operating system kernel that often contain bugs. 
Limiting kernel access to only frequently used \textbf{\textit{popular paths}}, which have previously been proven to contain fewer security bugs, would greatly reduce this risk. 
Therefore, if a container could run using only these paths, its security would be greatly enhanced. 

In this paper, we enhance the security of existing container designs by leveraging the popular paths metric and its security benefits. 
We designed and implemented the \textbf{\textit{Secure Logging System}}, which first systematically collects popular paths data by profiling popular containers, 
then use this data to create a \textbf{\textit{Secure Logging Kernel}} designed to test the feasibility of running containers only on those secure paths. 
If the application attempts to execute a line of code from an unpopular path, this kernel generates security logs and warning messages that alert users 
and help them decide whether the potentially dangerous code execution should be stopped. 
Our evaluation verified that more than 99.9\% of the time containers were able to run their default workload normally just using the popular paths. 
Furthermore, only 6\% of 50 Linux kernel CVE security vulnerabilities we examined were present in the popular paths on which these containers ran, 
which greatly reduced any potential risk of triggering kernel bugs. Our performance evaluation shows that 
containers running our Secure Logging Kernel only incur about 0.1\% runtime overhead on average, and around 0.5\% extra cost of memory space. 
This means the Secure Logging Kernel can improve security with little or no loss of efficiency. 