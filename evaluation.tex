\section{Evaluation}
\label{sec.evaluation}
The key idea of this paper is that by applying the popular paths metric onto an existing container design, LinuxKit, 
we can achieve stronger security and isolation for containers, while still maintaining functionality and performance. 
In other words, we want to demonstrate that using only the popular paths to run containers is feasible in real-world practice. 

To test this feasibility, we set out to answer the following questions: 

\begin{itemize}
\item Can real-world containers run only on the popular paths? (Section~{\ref{sec.evaluation.usability}})
\item Does restricting access only to the popular paths improve the security of  running containers? (Section~{\ref{sec.evaluation.security}})
\item What is the performance overhead of only using the popular paths? (Section~{\ref{sec.evaluation.performance}})
\end{itemize}

\subsection{Usability Evaluation}
\label{sec.evaluation.usability} 
We conducted experiments to test the feasibility of running real-world containers only using the popular paths. 
The goal here is to verify if users can still run their containers with their existing configuration and commands, 
using our secure logging kernel instead of the original Linux kernel. The secure logging kernel was created by our Secure Logging System.

\textbf{Experimental Setup.}
First, we collected the kernel trace data by running the 100 most popular (ranked by the number of user downloads) Docker containers from Docker Hub. 
We use this dataset as our popular paths training set. Next, our Secure Logging System used our training set data to instrument the Linux kernel, 
and generated our secure logging kernel. Finally, we ran another 100 popular Docker containers,  our testing set, on our security logging kernel. 

In our experiment, we ran the popular Docker containers inside of a LinuxKit version 0.2+ machine, which was built using Docker version 18.03.0-ce. 
Our host operating system was Ubuntu 16.04 LTS, with Linux kernel 4.13.0-36-generic. A QEMU emulator version 2.5.0 served as the local hypervisor. 

\textbf{Results.}
This is our results.

\subsection{Security Evaluation}
\label{sec.evaluation.security} 
We examined a list of 50 CVE kernel vulnerabilities, and verified that only three of them were triggerable in our LinuxKit popular paths kernel trace. 
This demonstrates that running containers using our secure logging kernel can help prevent kernel bugs from being triggered effectively. 

\textbf{Experimental Setup.}

\textbf{Results.}

\subsection{Performance Evaluation}
\label{sec.evaluation.performance} 
We evaluated both the run-time performance overhead and the memory space overhead of using our secure logging kernel. 

\textbf{Experimental Setup.}

\textbf{Results.}