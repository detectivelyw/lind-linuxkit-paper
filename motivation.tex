\section{Background and Motivation}
\label{sec.motivation}
To understand the central hypothesis of our work, that containers can be designed  to run with only limited access to the OS kernel, 
it is important to understand the design idea  on which it was built. Various metrics have been proposed over the last few decades as a way to predict where bugs may lie. 
Chou et al. \cite{Chou} focused on a particular part of the code, the device drivers, and argued that this component was the most likely place to find bugs. 
Ozment and Schechter \cite{Ozment} examined the age of code to see if that worked as a predictor of vulnerability in the OpenBSD \cite{OpenBSD}. 
Other researchers have also considered software engineering elements, such as the complexity of the code or how often a developer rewrites the code, 
as factors tied to bugs \cite{5560680, SAC10, Imtiaz2018TowardsDV, 4459644, Alenezi2015EvaluatingSM}. In terms of our current research, 
none of the above options have explored the feasibility of reducing the attack surface of the host OS kernel for containers. 
To enhance container security effectively, we need a security metric that could help identify where zero-day vulnerabilities are located in the kernel. 

The Popular Paths metric described in \cite{Lock-in-Pop} serves as a solid basis for us to protect containers from a vulnerable host OS kernel. 
It takes a quantitative approach, and evaluates security at the fine-grained  line-of-code level. 
It posits that lines of code in the kernel used to run popular software programs are less  likely  to contain security bugs. 
The study involved asking subjects to run  daily tasks, such as writing, spell checking, printing in a text editor, sending and receiving emails, and using an image processing program, 
on applications from the 50 most popular packages in Debian 7.0. Tests were conducted over a period of 5 calendar days for 20 hours of total use. 
In addition, the authors also had students use the workstation as their desktop machine for a one-week period to do their homework, develop software, communicate with friends 
and family, and so on. 
These combined experiments documented that only 2.5\% of zero-day bugs were found in these popular paths, demonstrating that this is indeed the safest part of the kernel.  

In addition, the study documented in \cite{Lock-in-Pop} that in practice, it is possible to design and build a virtualization system that can run large and complex legacy programs 
using only the popular paths. 
The system instantiated in the study was tested  on Tor and Apache, as well as on widely used tools, such as GNU Grep, GNU Wget, GNU Coreutils, GNU Netcat, and K\&R Cat. 
Results from the paper \cite{Lock-in-Pop} shows that the incurred performance overhead was around 2.5x to 5x, or about the same magnitude as existing state-of-the-art library OS systems. 

Most significantly, the study \cite{Lock-in-Pop} showed that leveraging the popular paths metric to design and build new architecture can lead to a more secure virtualization environment. 
This newer work explores the feasibility of utilizing this metric on an existing container system to improve security, and provide a solid standard for container security evaluation. 