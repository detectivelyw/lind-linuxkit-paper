\section{Background and Motivation}
\label{sec.motivation}
The concept of OS virtualization, where an operating system runs programs in an isolated space over a kernel, has been around for several decades \cite{OS-Level-Virtualization}. 
Yet,  the emergence of container programs \cite{Containers_and_VirtualMachines}, which run off the kernel's OS and therefore have a lower overhead, 
has spurred  growth in the popularity of these machines, as  Docker \cite{Docker}, LinuxKit \cite{LinuxKit}, and Kubernetes \cite{Kubernetes}, have seen wider adoption over the past few years. 
For example, Docker Hub \cite{DockerHub} holds more than two million container images , and the most popular ones have been downloaded by more than ten million users. 
Many users prefer using containers over traditional full-scale virtual machines such as VMware workstation \cite{VMWare-Workstation} and VirtualBox \cite{VirtualBox}, 
because containers allow them to test and develop their software programs with relatively low overhead. 
In addition, the idea of running their programs in a ``contained'' environment tends to make developers feel they have secured them against all threats. 

Existing container systems use several mechanisms to provide isolation and security to the programs running inside of them. 
Namespaces [8] were a modern Linux kernel feature, introduced between Linux kernel version 2.6.15 and 2.6.26, to provide a global system resource abstraction 
that helps achieve isolation between different containers. Docker and LXC [9] containers also used namespaces to provide isolation. 
Control groups [10] are another key Linux feature that enables accounting and limiting important system resources, such as memory, CPU, and disk I/O, to each container.  
In addition, this feature helps prevent denial-of-service attacks. Linux kernel capabilities [11] are used by a number of container systems, including Docker. 
Capabilities allow containers restrict permissions on specific resources, such as network and file system. 
Modern Linux kernels also have other kernel hardening systems, such as AppArmor [12], SELinux [13], GRSEC [14], PAX [15], etc., 
that may be leveraged by containers to provide some extra safety. GRSEC and PAX add safety checks at compile-time and run-time of the kernel. 
Docker ships a template that works with AppArmor, and also has SELinux policies working with Red Hat [16]. 
These kernel hardening systems are mostly access control mechanism that provides safety in the similar way that capabilities do.

While containers leverage existing mechanisms to try to improve security, there is yet one fundamental issue that is really challenging to address. 
The problem is, as acknowledged above,  that unlike ia full-scale virtual machine, the operating system kernel is shared among all containers and the host. 
This magnifies the damage a kernel vulnerability could cause. If a bug in the kernel was triggered from one guest container, it could result in a security problem for other containers, 
and even the host, breaking all the isolation that users expect to get from a container.

The threat of zero-day kernel vulnerabilities is the main motivation behind this work. 
This research initiative was launched to prevent zero-day kernel bugs from being triggered in containers, while still allowing containers to perform their desired workload. 
Having already established in a previous paper that   implementing the popular paths metric i could identify and neutralize the threat of these bugs [3] 
we theorized that container security can be enhanced in the same manner. 
To prove this hypothesis, we created a new design implementation for container security that is described in detail in the following section.