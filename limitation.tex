\section{Limitations}
\label{sec.limitation}
While our results and analysis indicate that our dataset was representative of the workload container applications perform, the sample size was somewhat limited and therefore, 
so was the amount of data gathered. Additional runs and a larger sample base could potentially capture the race-condition related kernel footprint more comprehensively, 
thus rendering more accurate data. 

Another factor that could have affected our results was the type of images we used. We limited our selection to official container images from Docker Hub, 
yet there are a number of open source project images we did not access. If we had run these additional images using the same methodology, 
it might have yielded a broader spectrum of results.

Our implementation is currently focused only on warning users whenever there is a potentially dangerous attempt to trigger an unpopular line of code. 
There are certainly more ways we could have explored to deal with risky attempts to reach unpopular lines. 
For example, we could have instrumented the kernel to try to exit the VM and block the container from executing more code to avoid any further security breach. 
Another possible strategy is to first monitor any attempt to access the unpopular lines, and then dynamically remove the unpopular code binary from the memory at runtime 
to prevent potential exploitation of bugs. We leave evaluation of these options as future work.

Lastly, all the research documented in this paper was conducted only with Linux kernel version 4.14.24. 
For future work, we plan to extend our system so that it can automatically work with other kernel versions. 
This will generalize the application of our work, and potentially help a lot more users secure their containers. 