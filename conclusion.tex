\section{Conclusion}
\label{sec.conclusion}
Widely used containers, such as Docker and LinuxKit, are not designed to protect against zero-day bugs found inside the kernel itself. 
As adding monitoring and auditing systems can increase the size and complexity of kernel code, a targeted strategy that effectively identifies the best places to audit is highly desirable. 
This is particularly true in container systems, which are designed to be lightweight. 
In this paper we introduced such a strategy that leverages previous findings \cite{Lock-in-Pop} that fewer zero-day kernel bugs are located in the most used kernel paths. 
We developed a methodology to systematically profile and identity these ``popular paths'' in widely-used container applications. 
With this data, we were able to create \textbf{UnPAK}, a modified Linux audit kernel that automatically logs access to the unpopular kernel code. 
We then developed a novel data structure called \textbf{SmashPAK} that catalogs and stores data on the location of this unpopular code. 
SmashPak can be queried by systems managers about specific CVE bugs and thus conduct informed forensic audits for potential exploits. 

We verified that conducting security audits using UnPAK created no loss of functionality for the containers. When auditing real-world container applications, 
UnPAK was able to operate with only minor increments (0.1\% on average) in runtime overhead, and only around 0.37\% extra cost in memory space. 
UnPAK tailors around 80\% of the kernel codebase, and identifies around 94\% of kernel CVE bugs. 
With our novel data structure SmashPAK, it was easy and efficient to store, update, and search for kernel logs. 
The logging data gathered by UnPAK and stored in SmashPAK can be searched in as little as a few milliseconds to learn if (and when) a CVE could have been triggered at a particular location in the container kernel. 