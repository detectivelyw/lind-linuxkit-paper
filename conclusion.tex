\section{Conclusion}
\label{sec.conclusion}
In earlier work, we found that running applications using only frequently used popular paths could improve the security of virtual machines. 
As a follow-up we wanted to test the feasibility of applying this popular paths metric to existing containers. 
To do so, we developed a methodology to systematically profile and obtain the popular paths data of widely-used container applications. 
Using this data, we were not only able to create UnPAK, a modified Linux kernel that could alert users whenever potentially dangerous unpopular lines are about to be triggered, 
but  also identify ways to improve security at various levels of granularity. By allowing users to improve container security with steps as simple as eliminating unused files, 
it means that those who can not do the more labor intensive work of eliminating bugs at the line of code level do not need to remain completely vulnerable. 
Security thus stops being an ``all or nothing'' proposition, but instead can be tailored to the security sensitivity of the application and the available resources of the user. 

We were also able to verify that, consistent with previous findings, the popular paths in the host kernel for frequently used Docker containers do contain fewer security vulnerabilities, 
and thus are inherently more secure. Equally important, usingUnPAK, we were able to prove that  these containers were able to run their default workload using the popular paths 
more than 99.9\% of the time, with only a negligible (less than 1\%) performance overhead based on its current action configuration. 
Compared with three other current kernel tailoring strategies, we demonstrate that UnPAK was an efficient solution in reducing the kernel attack surface, with relatively fewer limitations. 