\section{Introduction}
\label{sec.introduction}
Containers continue to grow in popularity, with one industry survey suggesting the technology could become a \$2.7 billion market by 2020 \cite{451-Research}. 
This widespread adoption of containers, such as Docker \cite{Docker} and LinuxKit \cite{LinuxKit}, can be attributed to several factors, 
including portability and elimination of the need for a separate operating system \cite{what-containers-do}. 
In addition, the perceived isolation and security they provide to the user programs running inside of them is reassuring to users looking to protect their data and operation. 
However, this perception may be false, as several recent incidents have demonstrated it is possible to trigger zero-day kernel bugs from inside of a container \cite{containers-kernel-bug-tcp, containers-runc-vulnerability}. 
Furthermore, since the kernel is the critical and privileged code shared by containers and the host system, exploitation of such bugs could lead to severe security problems, such as privilege escalation. 
The famous ``Dirty COW'' vulnerability that emerged in 2016, reported as CVE-2016-5195, allow attackers to escape from a Docker container and access files on the host system \cite{dirty-cow}. 
This vulnerability affected all of the Linux-based operating systems that used older versions of the Linux kernel, including Android. 
One analysis, conducted a year after it was first reported, found the bug in more than 1200 malicious Android apps, affecting users in at least 40 countries \cite{dirty-cow-impact}. 

The fundamental reason such exploits can occur is that existing containers are designed to directly access the underlying host operating system kernel on which they are run. 
This design, which makes containers more efficient and lightweight to use—features that make them so attractive to end-users— 
unfortunately exposes them to zero-day bugs within the kernel itself \cite{containers-kernel-bug-tcp}. 
To this point, there has been little or no effort to limit such contact, in part because there was no efficient way to identify where these bugs might be. 
If one knew which part of the kernel was free of zero-day bugs — or at the very least was less likely to host this threat— then restricting a container to touch only this part of the kernel would greatly improve security. 

Yet, targeting where the bugs are likely to be has been an elusive goal for many years. 
A number of researchers have promoted metrics \cite{Chou, Ozment} that claim to be able to predict the presence of bugs, but most have proven only minimally effective. 
That is, until two years ago when a study \cite{Lock-in-Pop} demonstrated a powerful correlation between the popularity of a kernel code path and its likelihood to contain a security flaw. 
Furthermore, the study \cite{Lock-in-Pop} demonstrated that these frequently used kernel paths can be leveraged to design and construct secure virtualization systems. 
Building on these results, we ask the question, can we apply the popular paths metric to  securing existing containers? 
And, if such an application is possible, could it provide stronger security and truer isolation than the existing design of container systems permits? 
To do so, we would first have to answer the question: Can applications in containers run without access to the whole kernel?

In this paper, we document our efforts to answer those questions by studying the feasibility of using the popular paths metric to secure the LinuxKit container toolkit. 
This entailed first, finding a way to identify the parts of the underlying kernel that are less likely to contain security bugs. We were able to demonstrate a systematic way to gather data on popular paths, 
and affirmed that this data is representative of hundreds of popular Docker containers from Docker Hub \cite{DockerHub}. 
Next, we produced a Secure Logging Kernel designed to log any attempt to access the unpopular paths, and to warn users away from code found in the less-used paths. 
The Secure Logging Kernel contributed to our study in two ways. Firstly, it provided usable data on how often applications used these risky paths, 
so we could determine whether such paths are essential to their functionality. 
And, secondly, it built into our instantiation of the modified kernel a way for users to detect potentially dangerous program behaviors and make decisions about whether or not to block executions in unpopular paths. 

An evaluation of our findings affirmed that popular Docker containers experience no loss of functionality when using just the popular paths. 
Indeed, for the official Docker containers’ default workload, the Secure Logging Kernel was able to run it smoothly. 
And our data shows that the popular paths were used more than 99.9\% of the time. In the case of less than 0.1\% of the time, unpopular lines were used, 
and were captured by the Secure Logging Kernel with security logs generated. 
Our security evaluation confirmed that using the popular paths to run Docker containers can effectively prevent most kernel bugs from being triggered, 
as only three of the 50 CVE kernel security vulnerabilities we checked for were found in those LinuxKit paths. 
And, in our performance evaluation, we found that running our Secure Logging Kernel only incurred about 1\% of runtime overhead, while the memory space overhead was only about 0.37\%. 
Thus, greater security could be achieved with very little perceived difference in operation efficiency or cost. 

In summary, we make the following contributions in this paper: 
\begin{itemize}
	\item We systematically identify and capture the ``popular paths'' data from Docker containers running in the LinuxKit VM. 
	\item We design and implement a Secure Logging System, that utilizes the popular paths data to create a modified Linux kernel (Secure Logging Kernel). This instrumented kernel outputs security warning messages whenever there is an attempt to reach and trigger unpopular paths. 
	\item Using the Secure Logging Kernel, we demonstrate that popular Docker containers were able to run their default workload normally all the time with negligible (less than 1\%) performance overhead. Popular paths were used more than 99.9\% of the time. Unpopular paths were reached only less than 0.1\% of the time, and were recorded by the Secure Logging Kernel with security logs to warn the users about potential malicious program behavior. 
\end{itemize}