\section{Introduction}
\label{sec.introduction}
Containers continue to grow in popularity, with one industry survey suggesting the technology could become a \$2.7 billion market by 2020 \cite{451-Research}.  
This widespread adoption of containers, such as Docker \cite{Docker} and LinuxKit \cite{LinuxKit}, can be attributed to several factors, 
including portability and elimination of the need for a separate operating system \cite{what-containers-do}. 
In addition, the perceived isolation and security they provide to the programs running inside of them is reassuring to users looking to protect their data and operations. 
However, this perception of security is actually a misinterpretation of what container security systems are designed to do. 
In essence, they defend against outside threats, rather than zero-day kernel bugs inside of a container \cite{containers-kernel-bug-tcp, containers-runc-vulnerability}.

Container access to the kernel also increases the chance of an exploit \cite{containers-kernel-bug-tcp}. 
The famous ``Dirty COW'' vulnerability that emerged in 2016, reported as CVE-2016-5195, allowed attackers to escape from a Docker container and access files on the host system \cite{dirty-cow}. 
This vulnerability affected all of the Linux-based operating systems that used older versions of the Linux kernel, including Android. 
One analysis, conducted a year after it was first reported, found the bug was being exploited in more than 1200 malicious Android apps, affecting users in at least 40 countries \cite{dirty-cow-impact}.  
Being able to detect and monitor vulnerabilities like this is important to container security and thus there is a growing demand for security auditing systems that can affirm if and when a vulnerability has been triggered.

However, such exploits are tricky to detect and monitor in container systems because, by design, all containers and the host system share the kernel as privileged code. 
This design makes containers more efficient and lightweight to use—features that make them so attractive to end-users. 
Adding a monitoring system to check for and store data on when and how kernel bugs have been triggered would  largely eliminate these benefits. 
What is needed is a more focused methodology that narrows the scope of what code within the kernel has to be monitored with high accuracy. 
Fortunately, a study published by Li et al. \cite{Lock-in-Pop} three years ago suggested a powerful correlation between kernel lines accessed by widely-used programs 
and their likelihood to contain security flaws. 
By appending security auditing only to code that is not popularly used, one could catch the overwhelming majority of security issues 
(97.5\% according to their measures), without overburdening the efficiency of the system.

Building on these results, this paper presents a multi-step methodology to help container developers and system administrators conduct forensic analyses when a security breach occurs. 
Based on captured code execution data from widely-used container applications, 
the forensic audit strategy helps to pinpoint the most likely places for an exploit by utilizing the output of an \textbf{UnPopular Audit Kernel (UnPAK)}, modified to log whenever a risky path is reached. 
To efficiently store, update, and index the forensic data gathered from the kernel logs, we designed a novel data structure called \textbf{SmashPAK}. 
A SmashPAK is effectively an index, in this case a list of logged unpopular basic blocks in the Linux kernel, that can be categorized into a searchable data structure. In our system, 
the UnPAK kernel emits SmashPAKs  at a specific frequency for each container. In doing so, it helps to answer the most pressing question, 
``Was this ever potentially exploited and if so, roughly when?'' and it does so with little to no overhead cost. 

Because the forensic strategy relies on a number of components, primarily UnPAK, we tested it against a number of performance, security criteria. 
We found that the initial popular paths data that guided the instrumentation of UnPAK’s logging function, could yield useful security information at three different levels of granularity—file, function, 
and line. This data testing at the line of code level tends to provide the most direct and accurate access to where risky code lies, being able to identify and log code at the file or function level can 
allow developers to conduct a fast audit at different granularity that may help some of them. At the file level, UnPAK showed that security audit at this granularity can be useful, 
as more than half of the kernel bugs reside in the unloaded and unused driver files. At the line level, over 90\% of the 50 CVE kernel vulnerabilities checked for were found in the unpopular kernel paths, 
showing that this is the right place for effective security auditing. And in general, one can track with reasonable accuracy which unpopular lines trigger a CVE across different kernel versions. 
We also found that UnPAK kernels are efficient and effective at capturing code path executions at the sites of later discovered CVEs, catching 94\% of CVE bugs we examined, 
making them effective tools for forensic auditing.

In performance tests, an evaluation of our findings affirmed that frequently used Docker containers experience no loss of functionality when run using just the popular paths. 
Indeed, we found these paths were used more than 99.9\% of the time to run the official Docker containers’ default workload. In addition, 
UnPAK was effective as a kernel tailoring strategy. In our tests, it tailored around 80\% of the kernel code and identified more than 90\% of CVE vulnerabilities. 
UnPAK also is a more flexible option as many of the existing strategies are tied to a specific application, or to addressing a particular subset of bugs. 
Lastly, we found that running UnPAK as currently designed only incurred about 1\% of runtime overhead, while the memory overhead was only about 0.37\%. 
Thus, valuable information that can prevent future kernel bug exploits can be obtained with very little perceived difference in operation efficiency or cost.

In summary, we make the following contributions in this paper:
\begin{itemize}
	\item We design a novel data structure called SmashPAK, which efficiently stores, updates, and indexes appropriate forensic logging data using a fixed amount of space. SmashPAK data structure allows system managers to learn in as little as a few milliseconds if (and when) specific bugs could have been triggered at a particular location.
	\item We develop a methodology to systematically capture the “popular paths” data for widely-used container applications, and verify that the technique works for the top 100 most downloaded Docker containers run inside of the LinuxKit VM.  We affirm that these popularity profiles can automatically update as new versions of the kernel are released.
	\item We use the popular paths kernel profile to implement UnPAK, a  tailored  Linux kernel that can generate logs for kernel forensic auditing using SmashPAK. 
	\item In performance tests, we demonstrate that auditing frequently-used Docker containers running their default workload using UnPAK incurs only a negligible performance overhead in runtime and memory. 
\end{itemize}